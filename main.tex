\documentclass[a4paper,12pt]{article}
\usepackage{graphicx}
\usepackage{caption}
\usepackage{subcaption}
\usepackage{cite}

\usepackage[english]{babel}

\usepackage[letterpaper,top=2cm,bottom=2cm,left=3cm,right=3cm,marginparwidth=1.75cm]{geometry}

\title{Navigating the landscape of Academic Writing}
\author{Mikael Montén}
\date{January 2024}

\begin{document}
\maketitle

\begin{abstract}
This paper explores the importance of academic writing, particularly in electrical engineering, statistics, and machine learning, and the challenges students face in mastering it. It discusses models for teaching academic writing, emphasizing the need for structured approaches from lecturers and students alike. The evolving nature of academic communication, potential shifts in writing forms, and unconventional avenues like expressive technical writing are also addressed. The conclusion underscores the shared responsibility of both educators and students in adapting to the evolving educational landscape to maintain the high standards that science requires.
\end{abstract}


\section{Introduction}

Academic writing stands as an essential tool for accurately conveying information and garnering credibility for one's work \cite{Source2}. Despite this, explicit education of academic writing is seldom seen in Swedish university courses, a trend seemingly mirrored internationally, as observed in the following peer-reviewed research articles. This lack of emphasis on academic writing poses challenges for students, hindering their ability to articulate their work effectively. The discrepancy is notably pronounced in technical fields, where attention to "soft values" such as writing is even scarcer compared to disciplines like i.e. behavioral or political sciences.

\section{Is academic writing relevant?}

Chrdileli and Shulzhenko \cite{Source2} argue that for electrical engineering students, the academic essay serves as the primary introduction to acquiring academic writing skills. This assertion extends to other technical disciplines as well like statistics and machine learning. Given the limited availability of academic essays, the bachelor thesis may serve as students' initial exposure to academic writing and its intricacies.

Enrolling in higher education not only aims to foster professional skills in a specific field but also emphasizes the ability to conduct research, think critically, and develop scientific projects \cite{Source2}. This aligns with students mastering the competence of presenting scientific results in written form. Additionally, the globalization of scientific communication necessitates proficient English written communication, thereby designating English as an "integral and essential" component of university studies \cite{Source2}.

Effective writing is not confined to academia; it is a valuable skill in industry, whether composing cover letters or business reports. Achieving this efficiently requires the capacity to convey information clearly, supported by good grammar, vocabulary, and techniques to convince the reader of the accuracy of the presented work \cite{Source2}. While many lecturers scrutinize grammatical errors and correct prepositions in written assignments, the ability to construct a logical argument through structured text deserves more attention to establish a coherent learning sequence \cite{Source2}.

Academic writing demands utmost precision and clarity, as highlighted, to accurately convey ideas and information. The proper use of articles and prepositions further enhances this capability \cite{Source3}. This poses a challenge for students lacking a solid English education in their native country, necessitating explicit education on the matter once one reaches higher education. Pronouns and articles, the most commonly used English words, may prove challenging for those whose native tongue has a severely different linguistic structure from English, requiring practice for mastery \cite{Source3}.

While recent AI advances can assist non-English speakers in preparing manuscripts and texts, too high of a reliance on external support could be detrimental to the scientific community in the long run \cite{Source3}.

\section{How can academic writing be taught?}

Despite the vast importance of academic writing, Warlizasusi et al. \cite{Source4} note that Islamic Education Management Postgraduate students often exhibit significant writing challenges in their research, prompting the development of a model for learning academic writing.

This model adopts a "4D development research model": Define, Design, Develop, Disseminate, and is further articulated into five steps \cite{Source4}:

1. Pre-writing: Generating ideas for writing through reading, brainstorming, clustering, etc.
  
2. Drafting: Writing while ensuring the correct arrangement of sentences and grammar.
  
3. Responding: Supervisors, lecturers, and possibly even students evaluate the authors' writing orally or in writing.
  
4. Revising: The student enhances the text based on the feedback received during the responding stage.
  
5. Evaluating: Lecturers conduct assessments, focusing on indicators of academic writing ability.

Implementing this entire process could assist students in structuring and refining their produced text. Potential issues may arise if students neglect thorough assessment after the pre-writing and drafting stages, possibly due to laziness or lecturers prioritizing their fields over teaching writing skills.

Another academic writing development model proposed by McGrath et al. \cite{Source5} employs a triadic approach, involving collaboration among three lecturers: an EAP (English for Academic Purposes) lecturer, a subject lecturer, and an academic developer. This resource-intensive model, while requiring more investment, proved effective in advancing the writing skills of subjects undergoing the training, demonstrating the benefits of "stepping out of silos to embed facets of writing development" \cite{Source5}.

\section{Should the educational system adapt?}

Or has the times changed? Sinclair \cite{Source1} delves into students' perspectives on authoritative writing in universities, contemplating potential shifts in future academic writing forms. While space and time reshape all other aspects of our world, the question arises: does academic communication remain exempt from these transformations?

Many academics deal with students complaining about the rigidity of citations \cite{Source1}, which might be due to students getting lazier - or are we experiencing a "transitional stage in academic writing globally, that perhaps requires an adaption of teaching practices"?.

The realm of student essays has transitioned from handwritten paper compositions to digital documents, offering possibilities like hyperlinking and annotations, yet \cite{Source1} suggests that these digital advancements are underutilized.

Warnock and Kahn \cite{Source6} explore unconventional avenues for academic writing development, introducing "expressive technical writing" (XTW). This extension of expressive writing, traditionally informal and self-directed, seeks to bridge the gap between fiction and diaries to the scientific community, especially within technical fields. Despite decades of research emphasizing the positive impact of mastering writing on engineering careers, \cite{Source6} notes that engineering students often undervalue writing, possibly due to a lack of awareness regarding the link between writing and thinking. The article provides examples of using XTW in software engineering to plan programming tasks, demonstrating how problem-solving can benefit from natural language and structured mind-maps. The implication is that such practices might encourage students in technical disciplines to engage more with writing. Could there be an added responsibility on lecturers to guide students in properly planning technical assignments?

Contrary to the notion that technical invention is less crucial as ideas are assumed to be pre-generated \cite{Source7}, a study involving 3 engineering students \cite{Source7} reveals that incorporating writing in the form of notes and lists enhances the application of acquired knowledge. Simultaneous "invention for writing, invention through writing, and technical invention" elevate results, underscoring the importance of not neglecting writing, whether formal or informal. Writing, as highlighted by Warlizasusi et al. \cite{Source4}, forms a foundational step in the stages of essay creation, prompting students to recognize it as essential alongside the technical aspects.

\section{Discussion and conclusion}

In conclusion, the significance of academic writing is obvious and extends beyond university studies, encompassing professional development and industry relevance. The global demand for effective written communication, particularly in English, emphasizes the need for explicit education in this skill. Various models, such as the 4D development research model and the triadic approach, offer structured approaches to teaching academic writing. As the educational landscape evolves, considerations for adapting teaching practices and exploring unconventional avenues, like expressive technical writing (XTW), become essential to address the evolving nature of academic communication and better prepare students for the challenges ahead.

The need for change is not however solely on lecturers. Yes, there should be focus on actually presenting relevant techniques to enhance learning and possibly having more assignments that relate to academic writing both for essays but also for structuring in general. Naturally however, it is always ones' personal responsibility to develop in areas one might be lacking in and therefore students simply must adhere to academic rules - as millions of people have done before. You can not continuously adapt the academic landscape for new generations, as it will eventually undermine the standards of science and with it, the future of humankind.


\bibliographystyle{plain}
\bibliography{references.bib}

\end{document}